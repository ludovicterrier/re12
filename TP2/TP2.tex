\documentclass[12pt,a4paper,notitlepage]{article}

\usepackage[utf8]{inputenc}

\usepackage[francais]{babel}\usepackage[T1]{fontenc}
\usepackage[cyr]{aeguill}
\usepackage{lmodern}
\usepackage{color}
\usepackage{boites}
\usepackage{fancybox}
\usepackage{listings}
\lstset{language=bash, basicstyle=\footnotesize, frame=shadowbox, rulesepcolor=\color{gris}}


\definecolor{gris}{gray}{0.75}
%\definecolor{bleup}{HTML}{258EE9}


%\renewcommand*\familydefault{\ttdefault} %% Only if the base font of the document is to be typewriter style
%\renewcommand{\rmdefault}{ptm}


\usepackage[
   pdfauthor={Ludovic Terrier & Arnaud Goulut},
   pdftitle={RE12 - TP2},
   ]{hyperref}
   
   
\usepackage[pdftex]{graphicx}

%\usepackage{titlesec}
%\titleformat{\section}[frame] {\normalfont} {\filright
%\footnotesize
%\enspace\textbf{\thesection}\enspace} {8pt} {\Large\bfseries\filcenter}

%% Je contrôle la taille de ma zone imprimée...
\usepackage{anysize}
%% ...en définissants les marges {gauche}{droite}{haute}{basse}
\marginsize{25mm}{15mm}{10mm}{15mm}

\begin{document}

\title{La configuration réseau sous Linux (2)}
\author{Arnaud Goulut et Ludovic Terrier}
\date{Avril 2010}
\maketitle


%\tableofcontents

\thispagestyle{empty}
\newpage

%%%%%%%%%%%%%%%%%%%%%%%%%%%%%%%%%%%  1ère page 


%%%%%%%%%%%%%%%%%%%%%%%%%%%%%%%%%%% 1ère partie
\section{Partie 1 : Les niveaux d'exécution}

\subsection{Paramétrage du service réseau}
On retrouve le paramétrage du service réseau dans le fichier : \texttt{/etc/init.d/network} :

\begin{lstlisting}
#! /bin/bash
#
# network       Bring up/down networking
#
# chkconfig: 2345 10 90
# description: Activates/Deactivates all network interfaces configured to \
#              start at boot time.
#
### BEGIN INIT INFO
# Provides: $network
# Should-Start: iptables ip6tables
### END INIT INFO\end{lstlisting}


\subsection{Exécution des scripts}

Les commandes liées à l'exécution des scripts sont situées dans le dossier \texttt{/etc/init.d/} qui sont les cibles des liens symboliques situées dans rcX.d.


\subsection{La commande chkconfig}



\section{Partie 2 : le super-serveur xinetd}
\subsection{Configuration de telnetd}
 
Ce qui peut vouloir dire que l'ensemble des autres ports sont dans le même VLAN par défaut.

\subsection{Les services à rattacher à xinetd}
Il est préférable de rattacher à xinetd des services qui sont peu utilisés, tel que des services d'accès à distance. En revanche, pour des services subissant de nombreuses connections (tel que web, ldap, messagerie) on n'utilisera pas xinetd.

\subsection{Filtrage d'accès}

Il existe deux moyens pour filtre l'accès au serveur telnet : 
\begin{itemize}
\item via les fichiers \texttt{/etc/hosts.allow} et \texttt{/etc/hosts.deny},
\item dans le fichier de configuration de chaque service.
\end{itemize}

\subsubsection{\texttt{host.deny} et \texttt{host.allow}}

\noindent Pré-requis : le fichier allow est prioritaire sur le fichier deny.\\

\begin{lstlisting}
# /etc/hosts.deny: list of hosts that are _not_ allowed to access the system.
ALL EXCEPT in.telnetd: 192.168.3.0
\end{lstlisting}
Ainsi, avec la ligne suivante on n'autorise personne (\texttt{ALL}) pour le service telnet (\texttt{in.telnetd}) avec pour exception le réseau local (\texttt{192.168.3.0}).


\subsubsection{fichier de configuration}


\subsubsection{permissif ou restrictif?}



La stratégie qui semble la plus sûre est celle utilisant un filtrage restrictif puisque l'on spécifie explicitement ce que l'on veut autoriser; donnant plus de contrôle sur les accès de la machine.


\section{Partie 3 : Serveurs d'accès distant}

\subsection{Attache à xinetd}

Pour le rattacher, il suffit de créer un fichier de configuration pour notre nouveau service, en ajoutant le paramètre :\\

\begin{lstlisting}
# default: on
service ssh
{
       flags		= REUSE
       socket_type	= stream
       wait		= no
       user		= root
       server		= /usr/sbin/sshd
       server_args	= -i
       log_on_failure	+= USERID
       disable		= no
}
\end{lstlisting}






\end{document}