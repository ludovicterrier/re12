\documentclass[12pt,a4paper,notitlepage]{article}

\usepackage[utf8]{inputenc}

\usepackage[francais]{babel}\usepackage[T1]{fontenc}
\usepackage[cyr]{aeguill}
\usepackage{lmodern}
\usepackage{listings}
\lstset{language=bash, basicstyle=\footnotesize, frame=shadowbox, rulesepcolor=\color{gris}}

\usepackage{amssymb}
\usepackage[table]{xcolor}
\definecolor{gris}{gray}{0.75}
\definecolor{bleup}{HTML}{258EE9}


%\renewcommand*\familydefault{\ttdefault} %% Only if the base font of the document is to be typewriter style
%\renewcommand{\rmdefault}{ptm}


\usepackage[
   pdfauthor={Ludovic Terrier & Arnaud Goulut},
   pdftitle={RE12 - TP2},
   ]{hyperref}
   
   
\usepackage[pdftex]{graphicx}

%\usepackage{titlesec}
%\titleformat{\section}[frame] {\normalfont} {\filright
%\footnotesize
%\enspace\textbf{\thesection}\enspace} {8pt} {\Large\bfseries\filcenter}

%% Je contrôle la taille de ma zone imprimée...
\usepackage{anysize}
%% ...en définissants les marges {gauche}{droite}{haute}{basse}
\marginsize{25mm}{15mm}{10mm}{15mm}

\begin{document}

\title{La configuration réseau sous Linux (2)}
\author{Ludovic Terrier et Arnaud Goulut}
\date{Avril 2010}
\maketitle


%\tableofcontents

\thispagestyle{empty}
\newpage

%%%%%%%%%%%%%%%%%%%%%%%%%%%%%%%%%%%  1ère page 


%%%%%%%%%%%%%%%%%%%%%%%%%%%%%%%%%%% 1ère partie
\section{Partie 1 : Les niveaux d'exécution}

\subsection{Paramétrage du service réseau}
On retrouve le paramétrage du service réseau dans le fichier : \texttt{/etc/init.d/network}

\begin{lstlisting}
#! /bin/bash
#
# network       Bring up/down networking
#
# chkconfig: 2345 10 90
# description: Activates/Deactivates all network interfaces configured to \
#              start at boot time.
#
### BEGIN INIT INFO
# Provides: $network
# Should-Start: iptables ip6tables
### END INIT INFO\end{lstlisting}


\subsection{Exécution des scripts}




\subsection{La commande chkconfig}



\section{Partie 2 : le super-serveur xinetd}
\subsection{Configuration de telnetd}
 
Ce qui peut vouloir dire que l'ensemble des autres ports sont dans le même VLAN par défaut.

\subsection{Les services à rattacher à xinetd}

\end{document}