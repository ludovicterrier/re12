\documentclass[12pt,a4paper,notitlepage]{article}

\usepackage[utf8]{inputenc}

\usepackage[francais]{babel}\usepackage[T1]{fontenc}
\usepackage[cyr]{aeguill}
\usepackage{lmodern}
\usepackage{color}
\usepackage{boites}
\usepackage{caption}
\usepackage{fancybox}
\usepackage{listings}
\usepackage{multicol}

\usepackage[T1]{fontenc}
\usepackage[scaled]{helvet}
\renewcommand*\familydefault{\sfdefault}

%\lstset{language=bash, basicstyle=\footnotesize, frame=shadowbox, rulesepcolor=\color{gris}, captionpos=b}

  \lstset{
         basicstyle=\footnotesize\ttfamily, % Standardschrift
         %numbers=left,               % Ort der Zeilennummern
         numberstyle=\tiny,          % Stil der Zeilennummern
         %stepnumber=2,               % Abstand zwischen den Zeilennummern
         numbersep=5pt,              % Abstand der Nummern zum Text
         tabsize=2,                  % Groesse von Tabs
         extendedchars=true,         %
         breaklines=true,            % Zeilen werden Umgebrochen
         keywordstyle=\color{red},
                frame=b,         
         keywordstyle=[1]{\itshape}{//},    % Stil der Keywords
         stringstyle=\color{white}\ttfamily, % Farbe der String
         showspaces=false,           % Leerzeichen anzeigen ?
         showtabs=false,             % Tabs anzeigen ?
         xleftmargin=5pt,
         framexleftmargin=1pt,
         framexrightmargin=5pt,
         %numbers=left,
         frame=toplines,
         framextopmargin=3pt,
       %  framexleftmargin=8pt,
         numberblanklines=false,
         %morecomment=[s][marron]{/*}{*/},
         %moredelim=*[s][\color{blue}]{/*}{*/}
         %morecomment=[s][marron]{/*-}{*/}},         
         framexbottommargin=5pt,
         captionpos=b,
         %backgroundcolor=\color{lightgray},
         showstringspaces=false      % Leerzeichen in Strings anzeigen ? 
 }

\DeclareCaptionFont{white}{\color{white}}
\DeclareCaptionFormat{listing}{\colorbox[cmyk]{0.43, 0.35, 0.35,0.01}{\parbox{\textwidth}{\hspace{10pt}#1#2#3}}}
\captionsetup[lstlisting]{format=listing,labelfont=white,textfont=white, singlelinecheck=false, margin=0pt, font={bf,footnotesize}}

\definecolor{gris}{gray}{0.75}
%\definecolor{bleup}{HTML}{258EE9}


%\renewcommand*\familydefault{\ttdefault} %% Only if the base font of the document is to be typewriter style
%\renewcommand{\rmdefault}{ptm}


\usepackage[
   pdfauthor={Ludovic Terrier & Arnaud Goulut},
   pdftitle={RE12 - TP5},
   ]{hyperref}
   
   
\usepackage[pdftex]{graphicx}

%\usepackage{titlesec}
%\titleformat{\section}[frame] {\normalfont} {\filright
%\footnotesize
%\enspace\textbf{\thesection}\enspace} {8pt} {\Large\bfseries\filcenter}

%% Je contrôle la taille de ma zone imprimée...
\usepackage{anysize}
%% ...en définissants les marges {gauche}{droite}{haute}{basse}
\marginsize{25mm}{15mm}{10mm}{15mm}

\begin{document}

\title{Mise en \oe uvre d'une infrastructure de ToIP}
\author{Arnaud Goulut et Ludovic Terrier}
\date{Juin 2010}
\maketitle


%\tableofcontents

\thispagestyle{empty}


 
%%%%%%%%%%%%%%%%%%%%%%%%%%%%%%%%%%% 1ère partie
\section{Etude théorique}
\subsection{SIP}
\subsection{messages}

\section{Architecture utilisée}
\subsection{configuration réseau}
\begin{figure}[!h]
\begin{center}
\includegraphics[height=4cm]{structure_reseau}
\caption{Structure du réseau}
\label{fig:da}
\end{center}
\end{figure}

\subsection{configuration téléphone}

\section{Fichiers de configuration SIP}
\subsection{sip.conf}
\subsection{extensions.conf}

\section{IAX} 
\subsection{pourquoi?}
A ce stade nous sommes en mesure de passer des appels entre divers téléphones situés sur le même réseau via le même serveur Asterisk. Or, parfois en entreprise, dans le cas de postes géographiquement séparés il peut être intéressant de pouvoir appeler des téléphones sur d'autres réseaux et utilisant un autre serveur de ToIP. Ceci est permit par le protocole IAX\footnote{Inter-Asterisk eXchange}. Pour les besoins du TP nous l'avons implémenté afin de pouvoir échanger des appels téléphoniques entre tous les groupes.

\begin{figure}[!h]
\begin{center}
\includegraphics[height=5cm]{structure_reseau_IAX}
\caption{Structure du réseau}
\label{fig:da}
\end{center}
\end{figure}

\subsection{fichier IAX.conf}

\section{Fichiers de traces}
\subsection{message INVITE (en détail)}
Un paquet d'INVITE SIP est composé de deux parties, un \textit{HEADER} et un  \textit{BODY}. C'est le détail de ces deux éléments que nous allons voir ici. 

\paragraph{}Pour ce faire nous avons émit un appel entre un hardphone (IP 192.168.3.129) vers un softphone X-lite (IP 192.168.3.3). Nous avons capturé à l'aide Wireshark l'ensemble des échanges depuis le serveur. Ainsi nous pouvons vous proposer ici le contenu du paquet INVITE capturé entre l'émetteur et le serveur Asterisk :
\subsubsection{HEADER}
\begin{lstlisting}[title=Contenu du HEADER d'un paquet INVITE de SIP]
 Via: SIP/2.0/UDP 192.168.3.129:5060;rport;branch=z9hG4bK31769CB150326F07D6EF3F1EE6061B5B
        From: user1 <sip:user1@192.168.3.1>;tag=1600664632
        To: <sip:102@192.168.3.1>
        Contact: <sip:user1@192.168.3.129:5060>
        Call-ID: 706088E9-6905-7FA9-B9A8-F002AC583FF6@192.168.3.129
        CSeq: 43045 INVITE
        Max-Forwards: 70
        Content-Type: application/sdp
        User-Agent: X-Lite release 1105d
        Content-Length: 312
\end{lstlisting}

\subsubsection{BODY}
\begin{lstlisting}[title=Contenu du BODY d'un paquet INVITE de SIP]

Session Description Protocol
            Session Description Protocol Version (v): 0
            Owner/Creator, Session Id (o): user1 4092070068 4092070070 IN IP4 192.168.3.129
            Session Name (s): X-Lite
            Connection Information (c): IN IP4 192.168.3.129
            Time Description, active time (t): 0 0
            Media Description, name and address (m): audio 8000 RTP/AVP 0 8 3 98 97 101
            Media Attribute (a): rtpmap:0 pcmu/8000
            Media Attribute (a): rtpmap:8 pcma/8000
            Media Attribute (a): rtpmap:3 gsm/8000
            Media Attribute (a): rtpmap:98 iLBC/8000
            Media Attribute (a): rtpmap:97 speex/8000
            Media Attribute (a): rtpmap:101 telephone-event/8000
            Media Attribute (a): fmtp:101 0-15
            Media Attribute (a): sendrecv
\end{lstlisting}

\subsection{message BYE} 
Voyons ici le contenu d'un paquet BYE émis par l'UA\footnote{User Agent} qui était à l'initiative de l'appel.

\begin{lstlisting}[title=Contenu d'un paquet BYE]
Session Initiation Protocol
    Request-Line: BYE sip:102@192.168.3.1 SIP/2.0
    Message Header
        Via: SIP/2.0/UDP 192.168.3.129:5060;rport;branch=z9hG4bK345EE39524AA335D955FEC4B5AC9A503
        From: user1 <sip:user1@192.168.3.1>;tag=1600664632
        To: <sip:102@192.168.3.1>;tag=as16bb0d2a
        Contact: <sip:user1@192.168.3.129:5060>
        Call-ID: 706088E9-6905-7FA9-B9A8-F002AC583FF6@192.168.3.129
        CSeq: 43046 BYE
        Max-Forwards: 70
        User-Agent: X-Lite release 1105d
        Content-Length: 0
\end{lstlisting}

\noindent \texttt{user@machine  chmod +x LdapAdminTool-4.6.1.x-Linux-x86-Install.bin \\
user@machine  ./LdapAdminTool-4.6.1.x-Linux-x86-Install.bin}
\bigskip






\end{document}